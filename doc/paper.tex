\documentclass{article}
\usepackage[margin=1.5in]{geometry}
\usepackage{minted}
\usepackage[bottom]{footmisc}
\usepackage{float}
\usepackage{stmaryrd}
\usepackage{cite}
\usepackage{amsmath}
\usepackage{amssymb}
\usepackage{subcaption}

\author{Harrison Goldstein - hjg42}
\title{CS 6115 Project}
\date{Fall 2017}
\newcommand{\coq}[1]{\mintinline{coq}{#1}}

\RecustomVerbatimEnvironment{Verbatim}{BVerbatim}{}

\begin{document}
\maketitle

\section{Introduction}
This semester I explored the space of provably correct regular expression
matchers. Regular expressions are an interesting focus of study because they are
deceptively simple. The languages that they recognize are very limited in scope,
and they can be fully defined by just a few constructors. That being said,
regular expressions can be extremely powerful in the right situations---they
have a broad range of applications, and implementing them is often more complex
than one would expect.

For my project, I focused on two different methods of implementing regular
expressions, and proved both correct. The first uses syntacitc Brzozowski
derivatives to extract an automaton from the regular expression, and the second
relies on putting the regular expression into ``standard form''. I had
originally planned to explore only the Brzozowski derivative approach, and I
made significant progress on that development during the middle third of the
semester. Unfortunately, as I was nearing completion of my correctness proof,
the same proofs were covered in class. Rather than throw away my work entirely,
I decided to include parts of that exploration in addition to work on regular
expression standardization.

In section \ref{defs}, I lay out the basic definitions that I use when proving
correctness. Section \ref{deriv} briefly covers my implementation and proof of
the Brzozowski derivative algorithm. In section \ref{standard}, I outline my
implementation and proof of the standardization procedure and of the associated
matching algorithm. Finally, section \ref{eval} presents a short evaluation of
my work, and section \ref{conclusion} concludes.


\section{Basic Definitions} \label{defs}
My Coq definition of regular expressions is shown in figure \ref{fig:re}. The
only interesting thing to note is that the type is parametrized by an alphabet,
\coq{T}.

\begin{figure}
  \centering
\begin{minted}{coq}
Inductive re (T : Type) : Type :=
| Void : re
| Ept : re
| Char : T -> re
| Alt : re -> re -> re
| Cat : re -> re -> re
| Star : re -> re.
\end{minted}
\caption{Regular expression definition.}
\label{fig:re}
\end{figure}
 In practice, all that matters about \coq{T} is that it has decidable equality,
which is enforced with the \coq{DecEq} type class. In the sequel, both \coq{T}
and \coq{DecEq T} will be assumed and not written.\footnote{The Coq development
uses the \coq{Section} and \coq{Context} mechanisms to do this formally.} The
other important definition is the denotational semantics for regular
expressions, shown in figure \ref{fig:in_re}. Note that the type \coq{str} is
simply an alias for \coq{list T}. This relation defines the \emph{meaning} of a
string \coq{s} being in the languge of a regular expression \coq{r}. In other
words
$$ \coq{[[ r ]] s} \qquad \equiv \qquad s \in \llbracket r \rrbracket $$
These definitions are fairly natural, and reflect the formal definitions of
regular expressions that are presented in theory literature.

\begin{figure}
  \centering
\begin{minted}{coq}
Reserved Notation "[[ r ]]" (at level 0).
Inductive in_re : re -> str -> Prop :=
| In_Ept : [[ Ept ]] []
| In_Char : forall c, [[ Char c ]] [c]
| In_Alt_left : forall r1 r2 s, [[ r1 ]] s -> [[ Alt r1 r2 ]] s
| In_Alt_right : forall r1 r2 s, [[ r2 ]] s -> [[ Alt r1 r2 ]] s
| In_Cat : forall r1 r2 s1 s2,
    [[ r1 ]] s1 -> [[ r2 ]] s2 ->
    [[ Cat r1 r2 ]] (s1 ++ s2)
| In_Star_empty : forall r, [[ Star r ]] []
| In_Star_cat : forall r s1 s2,
    [[ r ]] s1 -> [[ Star r ]] s2 ->
    [[ Star r ]] (s1 ++ s2)
where "[[ r ]]" := (in_re r).

Definition reg_eq r1 r2 := forall s, [[ r1 ]] s <-> [[ r2 ]] s.
Infix "[=]" := reg_eq (right associativity, at level 85).
\end{minted}
\caption{Denotational semantics for regular expressions.}
\label{fig:in_re}
\end{figure}

Also in figure \ref{fig:in_re}, I define regular expression equivalence in
terms of the denotational semantics. The Coq development includes a number of
proofs about this relation, including proving that it is a parametric relation.
This allows for rewriting of equivalent regular expressions, in limited cases.

Many of these definitions are inspired by or directly based on the work in
Lecture 20 of CS 6115. After that lecture, I went back and refactored my
existing code base to use the cleaner definitions presented by Professor
Morrisett. My first approach for the denotational semantics used Coq's
\coq{Ensemble} library, which provides tools for encoding sets directly, rather
than as predicates. The transition to the current definitions took a significant
amount of time, but resulted in much cleaner proofs.

\section{The Brzozowski Derivative Approach} \label{deriv}

\section{The Standard Form Approach} \label{standard}
In Bob Harper's 1999 paper titled Proof-Directed
Debugging,\cite{Harper:1999:PD:968578.968582} he presents a strawman algorithm
for matching regular expressions. The function, which he calls \coq{acc}, is
supposed to accept a string \coq{s} if and only if some initial portion of the
string is matched by the expression \coq{r}, and the rest of the string is
matched by the continuation, \coq{k}. This continuation-passing approach is
especially elegant because it allows for backtracking to happen naturally.

Unfortunately, the definition in figure \ref{fig:acc} is rejected by Coq. The
recursive call in the \coq{Star} case recurses on \coq{r}, not on \coq{r'}, and
thus Coq cannot determine a structurally decreasing argument. Even worse, there
might not even be a decreasing argument! In some cases, this \coq{acc} function
diverges and fails to determine anything at all.

\begin{figure}
  \centering
\begin{minted}{coq}
Fixpoint acc (r : re) (s : str) (k : str -> bool) :=
  match r with
  | Void => false
  | Ept => k s
  | Char c =>
    match s with
    | [] => false
    | d :: s' => if dec_eq c d then k s' else false
    end
  | Alt r1 r2 => acc r1 s k || acc r2 s k
  | Cat r1 r2 => acc r1 s (fun s' => acc r2 s' k)
  | Star r' => k s || acc r' s (fun s' => acc r s' k)
  end.
\end{minted}
\caption{The (incorrect) accept function.}
\label{fig:acc}
\end{figure}

\subsection{Standardization}
The paper presents a solution to this problem. It turns out that if regular
expressions are in a specific \emph{standard form}, the \coq{acc} function is
guaranteed to terminate. Furthermore, every regular expression can be put into
standard form without changing its meaning. Therefore, applying a standardizing
function and then applying \coq{acc} results in a function that terminates and
decides whether a string is accepted by a regular expression (and a
continuation).

The standardization function is defined in terms of two other functions, shown
in figure \ref{fig:fns}. The function $E(r)$ returns $\varepsilon$ if $r$
contains the empty string, and $\varnothing$ otherwise. (Note that $\oplus$ and
$\otimes$ are ``smart constructors'' for $+$ and $\cdot$ that simplify
expressions up to denotational equality.) The function $N(r)$ is slightly more
complicated. It satisfies the equation
$$ \llbracket N(r) \rrbracket = \llbracket r \rrbracket - \{\varepsilon\} $$
In the Coq code, these functions are called \coq{null} and \coq{nonnull}
respectively.

Putting these functions together, we can define the standard form as follows:
$$ \text{std}(r) = E(r) + N(r) $$

\begin{figure}
  \centering
  \begin{subfigure}[t]{0.45\linewidth}
    \begin{align*}
      E(\varnothing) &= \varnothing \\
      E(\varepsilon) &= \varepsilon \\
      E(c) &= \varnothing \\
      E(r_1 + r_2) &= E(r_1) \oplus E(r_2) \\
      E(r_1r_2) &= E(r_1) \otimes E(r_2) \\
      E(r^*) &= \varepsilon \\
    \end{align*}
    \caption{The \coq{null} function.}
  \end{subfigure}
  \begin{subfigure}[t]{0.45\linewidth}
    \begin{align*}
      N(\varnothing) &= \varnothing \\
      N(\varepsilon) &= \varnothing \\
      N(c) &= c \\
      N(r_1 + r_2) &= N(r_1) + N(r_2) \\
      N(r_1r_2) &= E(r_1)N(r_2) + N(r_1)E(r_2) \\
                     &\qquad + N(r_1)N(r_2) \\
      N(r^*) &= N(r)N(r)^*
    \end{align*}
    \caption{The \coq{nonnull} function.}
  \end{subfigure}
  \caption{$E$ and $N$.}
  \label{fig:fns}
\end{figure}

In the Coq development, I prove various facts about \coq{null}, \coq{nonnull},
and \coq{standardize}. The proof that every regular expression can be
standardized comes for free since \coq{standardize} is a function; the proof
that \coq{standardize} produces an equivalent regular expression not nearly as
automatic. The real effort in proving standardization correct comes when proving
the lemma \coq{nonnull_preserves_meaning}, which asserts that for any nonempty
string, \coq{s}, \coq{[[ r ]] s <-> [[ nonnull r ]] s}. The proof of the lemma
is almost 100 lines of ltac, and required proof of a number of related lemmas
and helper functions.

\subsection{Match Algorithm}
With a procedure for standardizing regular expressions in hand, it is possible
to write a function to determine if a given expression matches a given string.
First, I instrumented \coq{acc} with a fuel parameter. The function now takes an
extrac natural number argument which decreases on each recursive call. If the
parameter is ever zero, the function fails. Figure \ref{fig:match} shows the
definition of the \coq{match_re} function, as well as the correctness condition.

Note the existential quantification over \coq{n}---this says that there is some
amount of fuel such that the match is successful. If the proof of this theorem
is fully constructive, this implies that it is possible to \emph{compute} such
an amount of fuel, and thus write a total matching function. I hope to pursue
this line of reasoning in the future.

% TODO: Stuff aout proofs

\begin{figure}
  \centering
\begin{minted}{coq}
Definition match_re r s n := acc (standardize r) s is_nil n.

Theorem match_correct : forall r s,
  [[ r ]] s <-> exists n, match_re r s n = true.
\end{minted}
\caption{The matching function, and the associated correctness property.}
\label{fig:match}
\end{figure}

\section{Evaluation} \label{eval}

\section{Conclusion} \label{conclusion}

\bibliography{paper}{}
\bibliographystyle{plain}

\end{document}