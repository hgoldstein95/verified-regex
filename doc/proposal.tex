\documentclass{article}
\usepackage[margin=2cm]{geometry}
\usepackage{minted}
\usepackage{hyperref}
\usepackage[bottom]{footmisc}
\hypersetup{
  colorlinks=true,
  linkcolor=blue,
  filecolor=magenta,
  urlcolor=cyan,
}

\author{Harrison Goldstein - hjg42}
\title{CS 6115 Project Proposal}
\date{October 25, 2017}

\begin{document}
\maketitle

\section{Introduction}
This semester I was introduced to Syntactic Brzozowski derivatives of regular
expressions. In fact, I learned about it two unrelated times---once in Nate
Foster's Network PL class, and once by Dexter Kozen when he was a guest lecturer
in Theory of Computing.

I was so interested in the idea that I wrote a blog post about
it,\footnote{\url{http://harrisongoldste.in/languages/2017/09/30/derivatives-of-regular-expressions.html}}
and I wanted to do some work with it. I started implementing an algorithm for
matching strings using derivatives, and I thought it would be fun to prove it
correct. (My implementation of the algorithm can be found in appendix A.) I
quickly found that this wasn't as easy as I thought it would be.

\section{Minimum Goal}
My first goal with this project is to prove that my \verb|re_match| function
accurately decides whether a given regular expression matches a given string.
This will first require me to model correctness in some way, and then to show
that the algorithm is correct under that model. The plan is as follows:
\begin{enumerate}
  \item Define a relation \verb|denote : re -> Ensemble str -> Prop|, which
    relates regular expressions to the sets that they denote.
  \item Show, informally, that \verb|denote| is the correct denotational
    semantics for regular expressions.
  \item Prove the following theorem.
\begin{minted}{coq}
  Theorem re_match_correct :
    forall r l, denote r l -> (forall s, In str l s <-> re_match r s = true).
\end{minted}{coq}
\end{enumerate}
I hope to be able to do this without too much trouble, but the \verb|Ensembles|
package is missing a number of important lemmas (e.g. $A \cup B = B \cup A$,
etc.), so I will need to prove some auxiliary facts.

\section{Extensions}
I see a lot of ways for me to extend my project past just a correct matching
algorithm.
\begin{itemize}
  \item {\bf Verified \verb|grep|.} Since I will already have a verified regular
    expression matcher, I could theoretically implement some basic functionality
    of \verb|grep|. This would give me experience with code extraction and
    integrating extracted code with the outside world.
  \item {\bf Theory results.} With a model of regular expressions and their
    denotational semantics, I could prove some mathematical facts about regular
    languages.
  \item {\bf Extension to KATs.} It is well known that Brzozowski derivatives
    work with KATs as well. I could add tests to my language and prove more
    general facts about Kleene Algebras and KATs.
\end{itemize}

\pagebreak
\appendix
\section{Full Implementation}
\begin{minted}{coq}
Class DecEq (T : Type) := dec_eq : forall x y : T, {x = y} + {x <> y}.

Section Regex.
  Context {T : Type}.
  Context (dec_eq_ : DecEq T).

  Definition str := list T.

  Inductive re : Type :=
  | Phi : re
  | Char : T -> re
  | Alt : re -> re -> re
  | Con : re -> re -> re
  | Star : re -> re.

  Fixpoint contains_empty (r : re) : bool :=
    match r with
    | Phi => false
    | Char _ => false
    | Alt r1 r2 => contains_empty r1 || contains_empty r2
    | Con r1 r2 => contains_empty r1 && contains_empty r2
    | Star _ => true
    end.

  Fixpoint deriv (c : T) (r : re) : re :=
    match r with
    | Phi => Phi
    | Char c' =>
      match dec_eq c c' with
      | left _ => Star Phi
      | right _ => Phi
      end
    | Alt r1 r2 => Alt (deriv c r1) (deriv c r2)
    | Con r1 r2 =>
      if contains_empty r1 then
        Alt (Con (deriv c r1) r2) (deriv c r2)
      else
        Con (deriv c r1) r2
    | Star r => Con (deriv c r) (Star r)
    end.

  Fixpoint deriv_str (s : str) (r : re) : re :=
    match s with
    | [] => r
    | c :: s' => deriv_str s' (deriv c r)
    end.

  Definition re_match (r : re) (s : str) := contains_empty (deriv_str s r).
End Regex.
\end{minted}

\end{document}